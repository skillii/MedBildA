% **************************************************************************************************
% ** SPSC Report and Thesis Template
% **************************************************************************************************
%
% ***** Authors *****
% Daniel Arnitz, Paul Meissner, Stefan Petrik
% Signal Processing and Speech Communication Laboratory (SPSC)
% Graz University of Technology (TU Graz), Austria
%
% ***** Changelog *****
% 0.1   2010-01-25   extracted from report template by Daniel Arnitz (not ready yet)
% 0.2   2010-02-08   added thesis titlepage and modified layout (not ready yet)
% 0.3   2010-02-18   added TUG logo and statutory declaration
% 0.4   2010-02-18   moved the information fields below \input{./base/packages} (encoding...)
% 0.5   2010-03-02   added \ShortTitle to fix problems with long thesis titles
%                    added \ThesisType (makes the template suitable for MSc, BSc, PhD, ... Thesis)
%
% ***** Todo *****
% - Introduction/Usage
% **************************************************************************************************

% **************************************************************************************************
% basic setup
\newcommand{\DocumentType}{report} % "thesis" / "report"
\newcommand{\DocumentLanguage}{de} % "en" / "de"
\newcommand{\Twosided}{} % "twoside" / ""


% **************************************************************************************************
% template setup -- do not change these unless you know what you are doing!
\input{./base/packages}
\input{./base/layout_\DocumentType}
\input{./base/macros}
\graphicspath{{./drawings/}{./plots/}{./images/}}
% **************************************************************************************************
% ATTENTION: Make sure that makeindex is set to -s "./base/index.sty"
% **************************************************************************************************

% uncomment to get watermarks:
% \usepackage[first,bottom,light,draft]{draftcopy}
% \draftcopyName{ENTWURF}{160}


% **************************************************************************************************
% information fields

% general
\newcommand{\DocumentTitle}{Medizinische Bildanalyse UE}
\newcommand{\DocumentSubtitle}{Sheet 2}
\newcommand{\ShortTitle}{Sheet 2} % used in headers (keep short!)
\newcommand{\DocumentAuthor}{Ebner Thomas, Christoph Bichler}
\newcommand{\DocumentDate}{Graz, \today}
%    for thesis only (will be ignored for reports)
\newcommand{\ThesisType}{Master's Thesis}
\newcommand{\Organizations}{Signal Processing and Speech Communications Laboratory \\ Graz University of Technology \\[1cm] on behalf of \\ Some Company} % SPSC \\ TUG \\[1cm] on behalf of \\ A Nice Company
\newcommand{\Advisors}{Dipl.-Ing. Dr. Assoc.Prof. Klaus Witrisal \\ Dipl.-Ing. Paul Meissner} % Advisor 1 \\ Advisor 2 \\ ...
\newcommand{\Supervisors}{Univ.-Prof. Dipl.-Ing. Dr.techn. Gernot Kubin}

% revision number
\newcommand{\RevPrefix}{}
\newcommand{\RevLarge}{1}
\newcommand{\RevSmall}{0}

% confidential?
\newcommand{\ConfidNote}{}% {"confidential", "eyes only", ...}

% short command for vectors
\newcommand{\vect}[1]{\mathbf{#1}}


\begin{document}

%listingstyle:
\definecolor{orange}{rgb}{0.75,0.65,0}
\definecolor{gruen}{rgb}{0,0.5,0}
\definecolor{listinggray}{gray}{0.97}
\definecolor{listingshadow}{gray}{0.2}
\lstloadlanguages{Matlab}
\lstset{frame=shadowbox,
		rulesepcolor=\color{listingshadow},
		numbers=left,
		basicstyle=\scriptsize\ttfamily,
		numberstyle=\tiny,
		keywordstyle=\color{blue}\bfseries, % bold black keywords
		identifierstyle=, % nothing happens
		commentstyle=\color{gruen}, % comments
		stringstyle=\color{orange}, % typewriter type for strings
		showstringspaces=false,
		tabsize=4,
		backgroundcolor=\color{listinggray}
        }

% **************************************************************************************************
% titlepage
\input{./base/titlepage_\DocumentType}

% statutory declaration for theses
\ifthenelse{\equal{\DocumentType}{thesis}}{\input{./base/declaration}}{}


% **************************************************************************************************
% **************************************************************************************************
% user-defined part

\chapter{Task Description}

The task was to extract the center line of the vessels in the lung.
Therefore it was first necessary to detect the vessels, which have a tubular structure. The
next step was to find the center-line of the tubes. And in the last step
these center lines should be reconnected, to get a tree of the blood vessels.

\section{Tube Detection Filter}

This step was implemented according to Algorithm 3 in the lecture.
An image pyramid with 3 different scale levels was used.
For the tube detection we used a radius of 1 voxel.

\section{Center line Extraction and Reconstruction}

\subsection{Non maximum Suppression}

This step was also implemented according the lecture notes:

\subsection{Queue based reconnection}

We put all voxels with a value greater than \textit{threshold\_high} into a queue.

From each voxel in the queue we executed the following steps:

\begin{enumerate}
 \item Set \textit{skipped\_voxel} to zero.
 \item Add current voxel to the reconnected center line.
 \item Retrieve the tube direction of the current voxel (Eigenvector of smallest Eigenvalue of hessian matrix).
 \item Move 1 voxel along the Tube Direction.
 \item If value of the current voxel is greater than \textit{threshold\_low}:
  \begin{enumerate}
    \item Add all Voxels of the queue, containing the skipped voxel to the center line.
    \item Go to Step 1.
  \end{enumerate}
 \item If value of current voxel is smaller than \textit{threshold\_low}:
  \begin{enumerate}
   \item Increment \textit{skipped\_voxel}, and add the current voxel to a queue
   containing the skipped voxels.
   \item If skipped \textit{skipped\_voxel} is equal to 3:  abort.
   \item Go to step 2   
  \end{enumerate}
\end{enumerate}

The following thresholds were used:
\begin{itemize}
 \item \textit{threshold\_low}: 0.04
 \item \textit{threshold\_high}: 0.15
\end{itemize}

\chapter{Outcome of Experiment}

The resulting tree of the vessels was pretty impressive. The larger vessels were
detected and reconnected very well. At the outer regions of the lung, our algorithm had some problems
reconnecting ALL of the small vessels to a tree.
For perfectly reconnecting the smaller vessels, a more sophisticated algorithm(e.g using a spanning tree) is necessary.
Maybe also a dataset with a higher resolution would help to detect smaller vessels better.

We learned some new algorithms and how to use the ITK in a proper way.
It was also our first time we applied computer vision algorithms on a 3D dataset (volume).


\newpage


\section{Results (Output Images)}

\begin{figure}[h!]
 \centering
 \includegraphics[width=14.5cm,keepaspectratio=true]{./figures/max-white.png}
 \caption{Max Medialness over 3 different scales from 4 different viewpoints.}
 \label{fig:max}
\end{figure}


\begin{figure}[h!]
 \centering
 \includegraphics[width=15cm,keepaspectratio=true]{./figures/nonmaxima-white.png}
 \caption{Max Medialness after Non-Maximum Supression from 4 different viewpoints.}
 \label{fig:nonmaxima}
\end{figure}


\begin{figure}[h!]
 \centering
 \includegraphics[width=15cm,keepaspectratio=true]{./figures/centerline-white.png}
 \caption{Final Centerline after reconnection from 4 different viewpoints.}
 \label{fig:centerline}
\end{figure}


% **************************************************************************************************
% **************************************************************************************************

%\appendix
%\bibliographystyle{/.base/ieeetran}
%\bibliography{_bibliography}

% place all floats and create label on last page
\FloatBarrier\label{end-of-document}
\end{document}

